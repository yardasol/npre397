% use the answers clause to get answers to print; otherwise leave it out.
\documentclass[12pts, answers]{exam}
%\documentclass[12pts]{exam}
\RequirePackage{amssymb, amsfonts, amsmath, latexsym, verbatim, xspace, setspace}

% By default LaTeX uses large margins.  This doesn't work well on exams; problems
% end up in the "middle" of the page, reducing the amount of space for students
% to work on them.
\usepackage[margin=1in]{geometry}
\usepackage{enumerate}
\usepackage{hyperref}

% Here's where you edit the Class, Exam, Date, etc.
\newcommand{\class}{NPRE 397}
\newcommand{\term}{Summer 2017}
\newcommand{\assignment}{Nuclear Fuel Cycle Project}
\newcommand{\duedate}{2017.08.04}
%\newcommand{\timelimit}{50 Minutes}

\newcommand{\nth}{n\ensuremath{^{\text{th}}} }
\newcommand{\ve}[1]{\ensuremath{\mathbf{#1}}}
\newcommand{\Macro}{\ensuremath{\Sigma}}
\newcommand{\vOmega}{\ensuremath{\hat{\Omega}}}

% For an exam, single spacing is most appropriate
\singlespacing
% \onehalfspacing
% \doublespacing

% For an exam, we generally want to turn off paragraph indentation
\parindent 0ex

%\unframedsolutions
\usepackage{bibentry}
\begin{document} 

% These commands set up the running header on the top of the exam pages
\pagestyle{head}
\firstpageheader{}{}{}
\runningheader{\class}{\assignment\ - Page \thepage\ of \numpages}{Due \duedate}
\runningheadrule

\class \hfill \term \\
\assignment \hfill Due \duedate\\
%\begin{flushright}
%\begin{tabular}{p{5in} r l}
%\end{tabular}
%\end{flushright}
\rule[1ex]{\textwidth}{.1pt}

%%%%%%%%%%%%%%%%%%%%%%%%%%%%%%%%%%%%%%%%%%%%%%%%%%%%%%%%%%%%%%%%%%%%%%%%%%%%%%%%%%%%%
%%%%%%%%%%%%%%%%%%%%%%%%%%%%%%%%%%%%%%%%%%%%%%%%%%%%%%%%%%%%%%%%%%%%%%%%%%%%%%%%%%%%%

For \class, your total grade will be earned through a comprehensive project. It 
is intended to tie together ideas regarding reactor physics and fuel cycle 
analysis. This will culminate in an independent analysis of fresh and spent 
fuel isotopic compositions for current and advanced reactors. 
The project will be assessed as independent research work, much 
like a journal article undergoes peer review. I will be looking for : 

\begin{itemize}
\item Relevance
\item Technical Detail 
\item Analytic Rigor
\item Verifiability
\item Clarity
\item A Conclusion
\end{itemize}

This work will consist of three deliverables:
\begin{itemize}
        \item a work plan,
        \item a database of fresh and spent fuel compositions,
        \item and a final report. 
\end{itemize}

% ---------------------------------------------
\begin{questions}
\addpoints
% intro
\question[10] \textbf{Work Plan: Due 2017.05.30}

To help establish scope and milestones, the first step of the project will be a 
        work plan. Once you submit this plan, I will respond with feedback and 
        appropriate GitHub issues will be assigned. The plan should meet the 
        following guidelines:

\begin{itemize}
\item Minimum 500 words.
\item Maximum 1000 words.
\item Two columns.
\item Reasonable margins.
\item 10 pt font or larger.
\item State the problem to be approached.
\item Motivate the problem, explaining its relevance.
\item Summarize the current state of existing data. 
\item Describe the approach and methods you will take to generate the database.
\item Propose a prioritized list of reactor types that will be evaluated. 
\item For each reactor type, identify references for core design parameters.
\item Propose a schedule for the analysis.
\item Fully describe the data that will be delivered.
\end{itemize}

Please arrange at least one short meeting with me before the project plan
so that I can help you refine your schedule. I would be happy to provide feedback 
on a draft of your plan (once) before it is due.


\question[20] \textbf{Database: Due \duedate}
Prepare a database, for use with the Cycamore Reactor model in the Cyclus 
framework. The database should be in xml, json, or another Cyclus-compatible 
format. The database should contain fresh fuel compositions and spent fuel 
compositions for reactors of interest.

\question[60] \textbf{Final Report: Due \duedate}

Prepare a final document in the style of a journal article or conference 
proceedings. It should meet the following guidelines:

\begin{itemize}
\item Minimum 3000 words.
\item Maximum 10000 words.
\item Two columns.
\item Reasonable margins.
\item 10 pt font or larger.
\item State the problem that was approached.
\item Motivate the problem, explaining its relevance.
\item Comprehensively report and cite the current state of the art in the literature.
\item Describe the approach, methods, and other elements of your solution.
\item Describe in detail: the analysis, software, data, conclusions produced in this work.
\item For each reactor type evaluated, tabulate core design parameters.
\item Include publication quality graphs and figures.
\item Cite and provide data and code generated for this work sufficient to reproduce the conclusions.
\item Compare this result to previous results in the literature, reinforce the relevance of the work. 
\item Suggest future work.
\end{itemize}

\end{questions}


%\bibliographystyle{plain}
%\bibliography{}

\end{document}
