% use the answers clause to get answers to print; otherwise leave it out.
\documentclass[12pts, answers]{exam}
%\documentclass[12pts]{exam}
\RequirePackage{amssymb, amsfonts, amsmath, latexsym, verbatim, xspace, setspace}

% By default LaTeX uses large margins.  This doesn't work well on exams; problems
% end up in the "middle" of the page, reducing the amount of space for students
% to work on them.
\usepackage[margin=1in]{geometry}
\usepackage{enumerate}
\usepackage{hyperref}
\usepackage{color,soul}

% Here's where you edit the Class, Exam, Date, etc.
\newcommand{\class}{NPRE 397}
\newcommand{\term}{Spring 2025}
\newcommand{\assignment}{Nueutron Transport Project}
\newcommand{\duedate}{2025.05.07}
%\newcommand{\timelimit}{50 Minutes}

\newcommand{\nth}{n\ensuremath{^{\text{th}}} }
\newcommand{\ve}[1]{\ensuremath{\mathbf{#1}}}
\newcommand{\Macro}{\ensuremath{\Sigma}}
\newcommand{\vOmega}{\ensuremath{\hat{\Omega}}}

% For an exam, single spacing is most appropriate
\singlespacing
% \onehalfspacing
% \doublespacing

% For an exam, we generally want to turn off paragraph indentation
\parindent 0ex

%\unframedsolutions
\usepackage{bibentry}
\begin{document} 

% These commands set up the running header on the top of the exam pages
\pagestyle{head}
\firstpageheader{}{}{}
\runningheader{\class}{\assignment\ - Page \thepage\ of \numpages}{Due \duedate}
\runningheadrule

\class \hfill \term \\
\assignment \hfill Due \duedate\\ %\begin{flushright}
%\begin{tabular}{p{5in} r l}
%\end{tabular}
%\end{flushright}
\rule[1ex]{\textwidth}{.1pt}

%%%%%%%%%%%%%%%%%%%%%%%%%%%%%%%%%%%%%%%%%%%%%%%%%%%%%%%%%%%%%%%%%%%%%%%%%%%%%%%%%%%%%
%%%%%%%%%%%%%%%%%%%%%%%%%%%%%%%%%%%%%%%%%%%%%%%%%%%%%%%%%%%%%%%%%%%%%%%%%%%%%%%%%%%%%

For \class, your total grade will be earned through multiple scientific and 
technical paper reviews as well as contribution to a comprehensive project. It 
is intended to tie together ideas regarding technical writing, neutron 
transport, and numerical methods. 

The primary work will be to prepare \hl{\textbf{insert plan here}}.

The project will be assessed as independent research work, much 
like a journal article undergoes peer review. I will be looking for : 

\begin{itemize}
\item Relevance
\item Technical Detail 
\item Analytic Rigor
\item Verifiability
\item Clarity
\item A Conclusion
\end{itemize}

This work will consist of four deliverables:

\begin{itemize}
        \item a work plan,
        \item completion of the Stanford Writing in the Sciences MOOC,
        \item editorial reviews on at least two technical documents,
        \item and \hl{\textbf{final deliverable. coauthored paper with olek?}}.
\end{itemize}

% ---------------------------------------------
\begin{questions}
\addpoints
% intro
\question[10] \textbf{Work Plan: Due 2025.01.23}

To help establish scope and milestones, the first step of the project will be 
        a 
        work plan. This plan, developed under supervision of Graduate Research 
        Assistant Olek Yardas, will be delivered in the form of multiple GitHub 
        issues with descriptions and deadlines describing the work you intend 
        to do in the ARFC group. Start with completing the ARFC new member 
        \href{https://github.com/orgs/arfc/projects/37/views/1}{checklist.}

Please arrange at least one short meeting with Prof. Huff to discuss the project plan
so that I can help you refine your schedule. I would be happy to provide feedback 
on a draft of your plan (once) before it is due.

\question[20] \textbf{Writing in the Sciences Self-Study: Due \duedate}
        Watch the full set of videos hosted at
        \href{https://www.coursera.org/learn/sciwrite/}{Coursera}.

\question[20] \textbf{Editorial Reviews: Due \duedate}
Complete at least two editorial reviews of ARFC technical writing in 
        collaboration with colleages within the ARFC group. 

\question[60] \textbf{Final Deliverable: Due \duedate}
        \hl{Complete this section to describe the main accomplishment you 
        expect to achieve in spring semester.}

\end{questions}


%\bibliographystyle{plain}
%\bibliography{}

\end{document}
